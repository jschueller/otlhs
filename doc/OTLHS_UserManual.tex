% Permission is granted to copy, distribute and/or modify this document
% under the terms of the GNU Free Documentation License, Version 1.2
% or any later version published by the Free Software Foundation;
% with no Invariant Sections, no Front-Cover Texts, and no Back-Cover
% Texts.  A copy of the license is included in the section entitled "GNU
% Free Documentation License".
% Copyright 2014 EDF
%

%%%%%%%%%%%%%%%%%%%%%%%%%%%%%%%%%%%%%%%%%%%%%%%%%%%%%%%%%%%%%%%%%%%%%%%%%%%%%%%%%%%%%%%%%%
\section{User Manual}

This section gives an exhaustive presentation of the objects and functions provided by the \textit{otlhs} module, in the alphabetic order.

\subsection{GeometricProfile}

\begin{description}

\item[Usage:] \rule{0pt}{1em}
  \begin{description}
  \item \textit{GeometricProfile()}
  \item \textit{GeometricProfile(T0)}
  \item \textit{GeometricProfile(T0,c)}
  \item \textit{GeometricProfile(T0,c,iMax)}
  \end{description}

\item[Arguments:]  \rule{0pt}{1em}
  \begin{description}
  \item \textit{T0}: a NumericalScalar, initial temperature (10 by default)
  \item \textit{c}: a NumericalScalar, geometric factor, must be between 0 and 1 exclusive (0.95 by default)
  \item \textit{iMax}: an UnsignedInteger, number of iterations (2000 by default)
  \end{description}

\item[Value :] a GeometricProfile

\item[Details :]  \rule{0pt}{1em}
  \begin{description}
  \item GeometricProfile constructor
  \end{description}

\item \textit{operator()}
  \begin{description}
  \item[Usage :] \textit{operator()(it)}
  \item[Arguments :] an UnsignedInteger, iteration number
  \item[Value :] a NumericalScalar, the value of temperature at the given time
  \end{description}

\end{description}

\subsection{LHSDesign}
\begin{description}

\item[Usage:] \rule{0pt}{1em}
  \begin{description}
  \item \textit{LHSDesign(bounds, size, randomDesign)}
  \end{description}

\item[Arguments:]  \rule{0pt}{1em}
  \begin{description}
  \item \textit{bounds}: an Interval, bounds of variables
  \item \textit{size}: an UnsignedInteger, number of sample points
  \item \textit{randomDesign}: a Bool, \texttt{generate()} method builds a random LHS when true (which is the default), otherwise a centered LHS
  \end{description}

\item[Value :] an LHSDesign

\item[Details :]  \rule{0pt}{1em}
  \begin{description}
  \item LHSDesign constructor
  \end{description}

\item \textit{getBounds}
  \begin{description}
  \item[Usage :] \textit{getBounds()}
  \item[Value :] an Interval, which are bounds of variables
  \end{description}

\item \textit{getSize}
  \begin{description}
  \item[Usage :] \textit{getSize()}
  \item[Value :] an UnsignedInteger, sample size
  \end{description}

\item \textit{isCenteredDesign}
  \begin{description}
  \item[Usage :] \textit{isCenteredDesign()}
  \item[Value :] a Bool, true if centered designs are generated, false otherwise
  \end{description}

\item \textit{generate}
  \begin{description}
  \item[Usage :] \textit{generate()}
  \item[Value :] a NumericalSample, new design
  \end{description}

\end{description}

\subsection{LHSResult}
\begin{description}

\item[Usage:] \rule{0pt}{1em}
  \begin{description}
  \item \textit{LHSResult()}
  \item \textit{LHSResult(bounds, spaceFilling)}
  \item \textit{LHSResult(bounds, spaceFilling, restart)}
  \end{description}

\item[Arguments:]  \rule{0pt}{1em}
  \begin{description}
  \item \textit{bounds}: an Interval, specify bounds for each marginal
  \item \textit{spaceFilling}: a SpaceFilling, space filling criterion
  \item \textit{restart}: number of restarts performed during optimization
  \end{description}

\item[Value :] an LHSResult

\item[Details :]  \rule{0pt}{1em}
  \begin{description}
  \item LHSResult constructor, which is only called by optimization algorithms
  \end{description}

\item \textit{getBounds}
  \begin{description}
  \item[Usage :] \textit{getBounds()}
  \item[Value :] an Interval, which contains bounds for each marginal
  \end{description}

\item \textit{getNumberOfRestarts}
  \begin{description}
  \item[Usage :] \textit{getNumberOfRestarts()}
  \item[Value :] an UnsignedInteger, number of restarts performed during optimization
  \end{description}

\item \textit{getOptimalDesign}
  \begin{description}
  \item[Usage :] \textit{getOptimalDesign()}
  \item[Value :] a NumericalSample, contains the best LHS design
  \end{description}

\item \textit{getOptimalDesign}
  \begin{description}
  \item[Usage :] \textit{getOptimalDesign(restart)}
  \item[Arguments :] an UnsignedInteger, to specify a restart number
  \item[Value :] a NumericalSample, contains the best LHS design found during this specific restart
  \end{description}

\item \textit{getOptimalValue}
  \begin{description}
  \item[Usage :] \textit{getOptimalValue()}
  \item[Value :] a NumericalScalar, the criterion value for the optimal LHS design
  \end{description}

\item \textit{getOptimalValue}
  \begin{description}
  \item[Usage :] \textit{getOptimalValue(restart)}
  \item[Arguments :] an UnsignedInteger, to specify a restart number
  \item[Value :] a NumericalScalar, the criterion value for the optimal LHS design found during this specific restart
  \end{description}

\item \textit{getAlgoHistory}
  \begin{description}
  \item[Usage :] \textit{getAlgoHistory()}
  \item[Value :] a NumericalSample, contains informations gathered by optimization algorithm
  \end{description}

\item \textit{getAlgoHistory}
  \begin{description}
  \item[Usage :] \textit{getAlgoHistory(restart)}
  \item[Arguments :] an UnsignedInteger, to specify a restart number
  \item[Value :] a NumericalSample, contains informations gathered by optimization algorithm during this specific restart
  \end{description}

\item \textit{getC2}
  \begin{description}
  \item[Usage :] \textit{getC2()}
  \item[Value :] a NumericalScalar, the C2 value for the optimal LHS design
  \end{description}

\item \textit{getC2}
  \begin{description}
  \item[Usage :] \textit{getC2(restart)}
  \item[Arguments :] an UnsignedInteger, to specify a restart number
  \item[Value :] a NumericalScalar, the C2 value for the optimal LHS design found during this specific restart
  \end{description}

\item \textit{getPhiP}
  \begin{description}
  \item[Usage :] \textit{getPhiP()}
  \item[Value :] a NumericalScalar, the PhiP value for the optimal LHS design
  \end{description}

\item \textit{getPhiP}
  \begin{description}
  \item[Usage :] \textit{getPhiP(restart)}
  \item[Arguments :] an UnsignedInteger, to specify a restart number
  \item[Value :] a NumericalScalar, the PhiP value for the optimal LHS design found during this specific restart
  \end{description}

\item \textit{getMinDist}
  \begin{description}
  \item[Usage :] \textit{getMinDist()}
  \item[Value :] a NumericalScalar, the mindist value for the optimal LHS design
  \end{description}

\item \textit{getMinDist}
  \begin{description}
  \item[Usage :] \textit{getMinDist(restart)}
  \item[Arguments :] an UnsignedInteger, to specify a restart number
  \item[Value :] a NumericalScalar, the mindist value for the optimal LHS design found during this specific restart
  \end{description}

\end{description}


\subsection{LinearProfile}
\begin{description}

\item[Usage:] \rule{0pt}{1em}
  \begin{description}
  \item \textit{LinearProfile()}
  \item \textit{LinearProfile(T0)}
  \item \textit{LinearProfile(T0,iMax)}
  \end{description}

\item[Arguments:]  \rule{0pt}{1em}
  \begin{description}
  \item \textit{T0}: a NumericalScalar, initial temperature (10 by default)
  \item \textit{iMax}: an UnsignedInteger, number of iterations (2000 by default)
  \end{description}

\item[Value :] a LinearProfile

\item[Details :]  \rule{0pt}{1em}
  \begin{description}
  \item LinearProfile constructor
  \end{description}

\item \textit{operator()}
  \begin{description}
  \item[Usage :] \textit{operator()(it)}
  \item[Arguments :] an UnsignedInteger, iteration number
  \item[Value :] a NumericalScalar, the value of temperature at the given time
  \end{description}

\end{description}

\subsection{MonteCarloLHS}
\begin{description}

\item[Usage:] \rule{0pt}{1em}
  \begin{description}
  \item \textit{MonteCarloLHS(LHS, N)}
  \item \textit{MonteCarloLHS(LHS, N, spaceFilling)}
  \end{description}

\item[Arguments:]  \rule{0pt}{1em}
  \begin{description}
  \item \textit{LHS}: an LHS, initial design
  \item \textit{N}: an UnsignedInteger, number of random designs
  \item \textit{spaceFilling}: a SpaceFilling, space filling criterion
  \end{description}

\item[Value :] a MonteCarloLHS

\item[Details :]  \rule{0pt}{1em}
  \begin{description}
  \item With the first usage, \texttt{spaceFilling} is fixed to \texttt{SpaceFillingMinDist}
  \end{description}

\item \textit{generate}
  \begin{description}
  \item[Usage :] \textit{generate()}
  \item[Value :] an LHSResult, this instance contains the best design and some other values
  \end{description}

\end{description}

\subsection{OptimalLHS}
\begin{description}

\item[Usage:] \rule{0pt}{1em}
  \begin{description}
  \item \textit{OptimalLHS(lhs)}
  \item \textit{OptimalLHS(lhs, spaceFilling)}
  \end{description}

\item[Arguments:]  \rule{0pt}{1em}
  \begin{description}
  \item \textit{lhs}: an LHS, initial design
  \item \textit{spaceFilling}: a SpaceFilling, space filling criterion
  \end{description}

\item[Value :] an OptimalLHS

\item[Details :]  \rule{0pt}{1em}
  \begin{description}
  \item OptimalLHS constructor
  \end{description}

\item \textit{getLHS}
  \begin{description}
  \item[Usage :] \textit{getLHS()}
  \item[Value :] an LHS, initial design
  \end{description}

\item \textit{getSpaceFilling}
  \begin{description}
  \item[Usage :] \textit{getSpaceFilling()}
  \item[Value :] a SpaceFilling, space filling criterion
  \end{description}

\item \textit{generate}
  \begin{description}
  \item[Usage :] \textit{generate()}
  \item[Value :] an LHSResult, this instance contains the best design and some other values
  \end{description}

\end{description}

\subsection{PlotDesign}
\begin{description}

\item[Usage:] \rule{0pt}{1em}
  \begin{description}
  \item \textit{PlotDesign(data, bounds)}
  \item \textit{PlotDesign(data, bounds, Nx, Ny)}
  \item \textit{PlotDesign(data, bounds, Nx, Ny, title)}
  \item \textit{PlotDesign(result)}
  \item \textit{PlotDesign(result, Nx, Ny)}
  \item \textit{PlotDesign(result, Nx, Ny, title)}
  \end{description}

\item[Arguments:]  \rule{0pt}{1em}
  \begin{description}
  \item \textit{data}: a NumericalSample, which represents an LHS design
  \item \textit{bounds}: an Interval, specify bounds for each marginal
  \item \textit{Nx}: an UnsignedInteger, number of grid intervals along X-axis (default value = size of design)
  \item \textit{Ny}: an UnsignedInteger, number of grid intervals along Y-axis (default value = size of design)
  \item \textit{title}: a String, plot title
  \item \textit{result}: an LHSResult, returned by an optimization algorithm
  \end{description}

\item[Value :] a PlotDesign

\item[Details :]  \rule{0pt}{1em}
  \begin{description}
  \item PlotDesign constructor
  \end{description}

\end{description}

\subsection{SimulatedAnnealingLHS}
\begin{description}

\item[Usage:] \rule{0pt}{1em}
  \begin{description}
  \item \textit{SimulatedAnnealingLHS(lhs)}
  \item \textit{SimulatedAnnealingLHS(lhs, profile)}
  \item \textit{SimulatedAnnealingLHS(lhs, profile, spaceFilling)}
  \item \textit{SimulatedAnnealingLHS(design, bounds)}
  \item \textit{SimulatedAnnealingLHS(design, bounds, profile)}
  \item \textit{SimulatedAnnealingLHS(design, bounds, profile, spaceFilling)}
  \end{description}

\item[Arguments:]  \rule{0pt}{1em}
  \begin{description}
  \item \textit{lhs}: an LHS, initial design factory
  \item \textit{design}: a NumericalSample, an initial design
  \item \textit{bounds}: an Interval, bounds of design
  \item \textit{profile}: a TemperatureProfile, temperature profile
  \item \textit{spaceFilling}: a SpaceFilling, space filling criterion
  \end{description}

\item[Value :] a SimulatedAnnealingLHS

\item[Details :]  \rule{0pt}{1em}
  \begin{description}
  \item With the first and fourth usages, profile and spaceFilling are respectively setted to default \texttt{GeometricProfile} and \texttt{SpaceFillingMinDist}
  \item With the second and fifth usages, spaceFilling is setted to \texttt{SpaceFillingMinDist}
  \end{description}

\item \textit{generate}
  \begin{description}
  \item[Usage :] \textit{generate()}
  \item[Value :] an LHSResult, this instance contains the best design and some other values
  \end{description}

\item \textit{generate}
  \begin{description}
  \item[Usage :] \textit{generate(restart)}
  \item[Arguments :] an UnsignedInteger, the number of restarts
  \item[Value :] an LHSResult, this instance contains the best design and some other values
  \end{description}

\end{description}

\subsection{SpaceFilling}
\begin{description}

\item[Usage:] \rule{0pt}{1em}
  \begin{description}
  \item \textit{SpaceFilling(impl)}
  \end{description}

\item[Arguments:]  \rule{0pt}{1em}
  \begin{description}
  \item \textit{impl}: a SpaceFillingImplementation, a specific implementation of SpaceFilling
  \end{description}

\item[Value :] a SpaceFilling

\item[Details :]  \rule{0pt}{1em}
  \begin{description}
  \item SpaceFilling constructor
  \end{description}

\item \textit{evaluate}
  \begin{description}
  \item[Usage :] \textit{evaluate(sample)}
  \item[Arguments :] a NumericalSample
  \item[Value :] a NumericalScalar, value of the criterion evaluated on this sample
  \end{description}

\end{description}

\subsection{SpaceFillingC2}
\begin{description}

\item[Usage:] \rule{0pt}{1em}
  \begin{description}
  \item \textit{SpaceFillingC2()}
  \end{description}

\item[Value :] a SpaceFillingC2

\item[Details :]  \rule{0pt}{1em}
  \begin{description}
  \item SpaceFillingC2 constructor
  \end{description}

\item \textit{evaluate}
  \begin{description}
  \item[Usage :] \textit{evaluate(sample)}
  \item[Arguments :] a NumericalSample
  \item[Value :] a NumericalScalar, value of the C2 criterion evaluated on this sample
  \end{description}

\end{description}

\subsection{SpaceFillingMinDist}
\begin{description}

\item[Usage:] \rule{0pt}{1em}
  \begin{description}
  \item \textit{SpaceFillingMinDist()}
  \end{description}

\item[Value :] a SpaceFillingMinDist

\item[Details :]  \rule{0pt}{1em}
  \begin{description}
  \item SpaceFillingMinDist constructor
  \end{description}

\item \textit{evaluate}
  \begin{description}
  \item[Usage :] \textit{evaluate(sample)}
  \item[Arguments :] a NumericalSample
  \item[Value :] a NumericalScalar, value of the mindist criterion evaluated on this sample
  \end{description}

\end{description}

\subsection{SpaceFillingPhiP}
\begin{description}

\item[Usage:] \rule{0pt}{1em}
  \begin{description}
  \item \textit{SpaceFillingPhiP(p)}
  \end{description}

\item[Arguments:]  \rule{0pt}{1em}
  \begin{description}
  \item \textit{p}: an UnsignedInteger (by default, 50)
  \end{description}

\item[Value :] a SpaceFillingPhiP

\item[Details :]  \rule{0pt}{1em}
  \begin{description}
  \item SpaceFillingPhiP constructor
  \end{description}

\item \textit{evaluate}
  \begin{description}
  \item[Usage :] \textit{evaluate(sample)}
  \item[Arguments :] a NumericalSample
  \item[Value :] a NumericalScalar, value of the PhiP evaluated on this sample
  \end{description}

\end{description}

